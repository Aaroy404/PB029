% Třída dokumentu
\documentclass[a4paper]{article}
% Preambule
%% Balíčky
\usepackage[czech]{babel}
\usepackage[T1]{fontenc}
\title{Značkování textu v \LaTeX u}
\author{Adamko}
\date{6. 9. 2023}
% Tělo dokumentu
\begin{document}
\maketitle

\begin{abstract}
V tomto článku si ukážeme, jak můžeme v \LaTeX u vyznačit strukturu dokumentu.
Po přečtení zdrojového kódu článku bude čtenář schopný připravovat vlastní
dokumenty v \LaTeX u.
\end{abstract}

Toto je minimální \LaTeX ový dokument. Tento dokument sestává z několika odstavců.

Odstavec obsahuje jednu nebo více vět, které mají tvořit určitý logický celek.
Programátoři často dělí programy do samostatných modulů, které řeší jednu úlohu
a jsou minimálně závislé na ostatních modulech. Podobně můžeme přemýšlet o odstavcích,
které by měly být samoobsažné a pochopitelné i bez odkazů na okolní odstavce.


Toto je další odstavec, který je oddělený od předchozího odstavce dvěma prázdnými řádky.

\tableofcontents

\section{Členění dokumentu}

Odstavce členíme do sekcí a podsekcí. Každá sekce by měla být uvozena alespoň jedním
odstavcem textu, který čtenáře seznámí s jejím obsahem. Nadpis sekce, po kterém okamžitě
následuje nadpis podsekce se považuje za chybu.

V této sekci si vytvoříme dvě podsekce pro ilustraci členění dokumentu.

\subsection{První podsekce}

Nadpisy (pod)sekcí udávají strukturu textu, ale nejsou přímou součástí textu.
Porozumění textu by tedy nemělo záviset na nadpisech. Formulace ,,v této sekci se
budeme zaobírat tématem z nadpisu sekce`` a podobné se považují za chybu.

Toto je první podsekce našeho dokumentu.

\subsection{Druhá podsekce}

Toto je druhá podsekce našeho dokumentu.

\end{document}