% Třída dokumentu
\documentclass[a4paper]{article}
% Preambule
%% Balíčky
\usepackage[czech]{babel}
\usepackage[T1]{fontenc}
%% Doplňte metadata
% Tělo dokumentu
\begin{document}
\section{Formátování textu}

    Formátování textu je postup, při kterém dojde k logickému rozčlenění textu, neboli k jeho zorganizování do logických celků. Formátování textů se provádí v celé řadě oborů, důležitý je ale druh díla, který má vzniknout (např. výpisky, webová stránka, román, báseň). Každý druh díla má pak svoje zažité specifikace formátování v tom či onom kulturním prostředí.

\section{Základní struktura} 

I zde závisí na druhu díla, které má vzniknout. Strukturalizace textu většinou daný typ díla uvozuje a zároveň dílo zpřehledňuje pro čtenáře či robota (například prohlížeč či čtečka pro slepce). Při strukturalizaci textu lze rozlišit několik základních objektů. Jejich vizuální podoba může být různá, funkčně jsou si však velmi blízké. Jedná se o:

\begin{itemize}
    \item odstavce - člení text do základních tematických celků;
    \item nadpisy a podnadpisy - pojmenovávají odstavce a skupiny odstavců;
    \item výčty a seznamy - jedná se o heslovité položky řazené buď číselně, v bodech nebo terminologicky.

\end{itemize}
Podstatou hypertextu jsou i odkazy odkazující na jinou stránku v případě tzv. hypertextových odkazů. Jinak zde odkazy plní stejnou funkci jako v běžném textu - odkazují na jiné části té samé stránky či dokumentu; odkazují na jiná textová či vizuální média;

\end{document}
