% Začátek preambule
\documentclass{beamer}
\usepackage[resetfonts]{cmap}
\usepackage{lmodern}
\usepackage[czech]{babel}
\usepackage[utf8]{inputenc}
\usepackage[T1]{fontenc}
\title{Titulek prezentace}
\subtitle{Podtitulek prezentace}
\author{Jméno autora}
% Konec preambule

% Začátek dokumentu
\begin{document}
    \frame{\maketitle}
    \frame{\tableofcontents}

    \section{První část prezentace}
    \subsection{Základní vlastnosti čísel}
    \begin{frame}
        \frametitle{Prvočísla}
        \framesubtitle{Definice a příklady}
        
        % Příklad byl převzat z dokumentace třídy `beamer`.
        \begin{definition}
            \alert{Prvočíslo} je celé číslo, které má právě dva dělitele.
        \end{definition}
        \begin{example}
            \begin{itemize}
                \item 2 je prvočíslo (dva dělitele: 1 a 2).
                \item 3 je prvočíslo (dva dělitele: 1 a 3).
                \item 4 není prvočíslo (\alert{tři} dělitele: 1, 2, a 4).
            \end{itemize}
        \end{example}
        \mode<presentation>{Tento text se zobrazí pouze v režimu prezentace.}
        \mode<article>{Tento text se zobrazí pouze v režimu článku.}
    \end{frame}
\end{document}
% Konec dokumentu
